\documentclass[ProjectRequirements.tex]{subfiles}
\begin{document}

\bigskip
\pagebreak
\section{\textsc{\Large Specific Requirements}}
	%%%%%%%%%%%%%%%
	%MARKETING REQUIREMENTS
	%%%%%%%%%%%%%%%
	\subsection{Market Requirements}
	\begin{enumerate}%[itemsep=-1ex]
		\item The system should be able to have a connection to a drone at 900 MHz up to a distance of 20 miles.
		\item The system should be able to track the drone by steering the phased array.
		\item The system should be able to be mounted on the roof of a car.
		\item The system should be able to connect to at most 4 drones at the same time.
		\item The system should be able to transmit 5 user commands per second.
		\item The system will not overheat.
		\item The system should be able to withstand severe weather.
		\item The system should be radio agnostic.
		\item The system will follow FCC regulations.
		\item The system will not require any extra power connections.
		\item The user can send controls and power to the system easily.
	\end{enumerate}
	%%%%%%%%%%%%%%%
	%ENGINEERING REQUIREMENTS
	%%%%%%%%%%%%%%%
	\subsection{Engineering Requirements}
	\begin{center}
	\begin{tabular}{| p{2.3cm} | p{5cm} | p{7cm} |} \hline
		Marketing Requirements & Engineering Requirements & Justification \\ \hline \hline
		3 & The phased array dimensions will not exceed L x W x H & The phased array needs to be small enough to possibly be mounted on a car. \\ \hline
		1,2,4 & The elevation should be from 0$^{\circ}$ to 30$^{\circ}$ & The drone will fly quite low and 20 miles away, the phased array needs to be able to communicate at low elevations.\\\hline
		1,2,4 & The azimuth should be from 360$^{\circ}$ & The drone can be flying anywhere around the phased array, so it needs to be able to transmit in a full circle. \\\hline
		6 & The system will have temperature monitoring. & To detect if the system is overheating or not, there will need to be temperature monitoring. \\\hline
		1,2,3 & System reconnects to drone if disconnected in anyway. & If the car drives through a tunnel, connection may be lost, the system will have to reconnect. \\\hline
		1,3,4  & The phased array steers at 100Hz. & To keep the phased array tracking the drone, it needs to update consistently. \\\hline
		8 & Will support the use of any generic SDR & To give the user more functionality, the user will want to use any SDR they have available.\\\hline
		8 & The system will support any packet scheme the user inputs. & Users will want to use their own packet scheme to communicate with their drones.\\\hline
		1,3,9 & The system must function between 902MHz and 928MHz. & The drone supports frequency switching between 902MHz and 928MHz, so the antenna must support it.\\\hline
		 9 & The output power of the antenna cannot exceed 36dBm. & FCC regulations define that 36dBm is the maximum amount of power output allowed at these frequencies.\\\hline
		10,11 & The system will be powered by PoE. & To keep wiring minimal and user friendly, having power over ethernet is important. \\\hline
		11 & All controls will be transferred over ethernet. &  Likewise with the power, keeping one cable to transmit both power and controls keeps the system clean.\\\hline
		1,2,4 & The system will be able to point towards the drone with only the GPS coordinates. & The general GPS coordinates of the drone will be known.    \\\hline
		1,2,5 & The system will have a 1Hz "heartbeat" to maintain a connection to the drone. & The system needs to maintain a connection to the drone to continue receiving and transmitting data.    \\\hline
	\end{tabular}
	\end{center}

\end{document}
