\documentclass[11pt]{article}
\usepackage{setspace}
\usepackage{hyperref}
\usepackage[utf8]{inputenc}
\usepackage[T1]{fontenc}
\usepackage{fullpage}
\usepackage{amsmath}
\usepackage{amssymb,amsfonts}
\usepackage[all,arc]{xy}
\usepackage{enumerate}
\usepackage{mathrsfs}
\usepackage{bbm}
\usepackage{graphicx}
\usepackage{subfiles}
\usepackage{setspace}
\usepackage{float}
\usepackage{subcaption}
\usepackage{verbatim}
\usepackage{listings}
\setlength{\footskip}{40pt}
\graphicspath{{Figures/}}
\numberwithin{figure}{section}
\setlength\parindent{0pt}

\begin{document}
%\doublespacing
\begin{center}
	\textsc{\Large Product Description}\\
	\hrulefill 
\end{center}

\section{\textsc{Team Members}}
	Matthew Arendall, Bryan DiLaura, Ariel Hoffman, Andrew Kee, Ryan Montoya, Jonathan Quinn
	
\section{\textsc{Team Name}}
	Phase Factor
	
\section{\textsc{What user problem is your product/device trying to solve?}}
	\textbf{High Priority Problems}  \\ 
	As drone applications increase, there is greater demand for well-engineered base stations. Currently, base stations typically use high frequency directional antennas that require mechanical steering to control the beam. There are a variety of problems associated with this setup. Firstly, the current technology is susceptible to more free space loss because of the higher frequency. Secondly, the system is more complex and less maintainable because of the mechanical components. We spoke to some members in the field and they explained some of the problems associated with mechanical parts which we have reproduced below:
	\begin{itemize}
		\item A mechanically steered antenna is not suitable for mobile applications.
		\begin{itemize}
			\item Often times, a user will want to follow and control his drone from a car. This requires having an antenna capable of being moved around. The current technology is not conducive to this because mechanical parts require complicated control systems in order to maintain a stable directionality.
		\end{itemize}
		\item A mechanical antenna limits slew rate.
		\begin{itemize}
			\item Mechanical components are, by nature, limited in speed. Getting a component moving at the appropriate speed without tearing itself apart can be a huge challenge. This is particularly challenging for users who might be interested in controlling multiple drones at once. Mechanically, that would require an antenna moving quickly back and forth to maintain a lock. This constant motion not only degrades the physical components, but does not provide the required speeds.
		\end{itemize}
		\item Mechanical steering requires motors and motors create strong magnetic fields
		\begin{itemize}
			\item The RECUV lab could not accurately point their antenna because the fields produced by the motors interfered with the ground station’s magnetometer
		\end{itemize}
	\end{itemize}
		
	Furthermore, even though the current practice is to try to direct the beam using mechanical means, the beam is still not very directional: it puts out signal in a large area. This makes the signal a security concern. Another ground receiver can potentially snoop on the communication channel and obtain sensitive information.  This can be a problem for customers working in a wartime scenario in which information can mean the difference between life and death.  \\

	Nonetheless, people put up with these problems because the base station needs to be able to transmit over a large area because the drone could be in a wide range of places. \\


	\textbf{Medium Priority Problems} \\
	Standard communication protocols have limited bandwidth for communication and scientific data.  Applications that require large amounts of data consume significant bandwidth and therefore are limited by the carrier frequency band.  Operating on a single channel requires multiplexing data and command/control signals, further increasing the complexity and bandwidth requirements of the system.  This can be a problem for customers interested in transmitting telemetry and wifi to their drone at the same time.  \\
	
	Another problem for people who want to work with high reliability drones is that sometimes links go down.  If an antenna stops working, it is important to be able to reestablish a link some other way.  \\
	
	\textbf{Low Priority Problems} \\
	If a user wants to be able to manage multiple drones at once, one option is to use multiple antennas.  This allows people to send the necessary bandwidth of data.  However, this solution can be expensive because it costs a lot to make multiple antennas and create independent tracking systems.  An alternative option is to use a mechanical antenna director and write firmware to allow the antenna to multiplex.  This is a problem because it is unreliable and slow due to its dependence on mechanical components.  Both of these problems are prohibitive to users trying to do research.
	

	
\section{\textsc{What is our proposed solution?}}
	\textbf{High Priority Solutions}  \\
	We propose using a phased array to solve these problems.  A phased array is an array of antennas that allows us to create a highly directive beam pattern.  This allows the user to transmit over a larger distance using less power when compared to a single dipole transmitting over the same distance.  We will be using a frequency of 900MHz because it has a relatively low attenuation in air.  This will allow for the signal to reach farther distances than higher frequencies.  \\
	
	The range of a phased array is significantly longer than an omnidirectional antenna due to directivity.  This will enable the base station to receive and transmit to the drone antenna at long ranges even if the drone only supports a standard dipole or low power signal.  Furthermore, the phased array is capable of tuning itself to receive a certain phase more effectively.  Assuming the phased array can keep a lock on the drone, this means that it will be able to receive lower power signals from the drone, which addresses the battery problem.  \\
	
	To address the concern of data security, the high directivity of the signal allows it to point to a very small area, thereby reducing the possibility that another person would be able to intercept the signal.\\
	
	\textbf{Medium Priority Solutions} \\
	The data and telemetry multiplexing problem can be addressed by using two isolated frequency bands.  We propose using two phased arrays stacked on top of one another.  This system would be capable of using both the 900MHz and 2.4GHz bands concurrently.  Two frequency bands gives the system a much greater bandwidth.  Additionally, using two isolated bands allows separation of data and command/control.  This reduces the complexity of the communications protocol.  A 2.4GHz phased array should be relatively simple to create after designing a 900MHz antenna because these systems can be scaled to match the frequencies.  We propose to use a 2.4 GHz channel for data as it has a greater bandwidth and is well suited for large data throughput, and to use a 900 MHz channel for command/control as it is long range.  \\
	
	\textbf{Low Priority Solutions} \\
	In order to solve the multiple drone problem, we propose using a high speed microcontroller to manage our phased arrays.  Because the phased array is capable of redirecting the beam electronically, it solves the mechanical issues.  Because we will be using a high frequency chip, we will be able to keep a lock on multiple drones at once.  Finally, because we will be able to package this functionality into a single product, it will be cheaper, more portable, and more maintainable.

\section{\textsc{Who needs this? Who will benefit from this?}}
There are three major beneficiaries: 
	\begin{enumerate}
		\item \textbf{Researchers} -- the drone is intended as a lightweight surveillance system.  The drone needs to be able to travel significant distances in potentially hard to reach areas and transmit data back to the base.  Our phased array project helps improve the functionality of the drone by increasing the range and reducing the weight of the drone.  The drone can be used for weather analysis, video streaming, or relaying information.
		\item \textbf{The autonomous vehicle market} -- most people within the market are using standard dipole antennas and are dealing with all of the shortcomings as a result (discussed earlier).  A phased array would be advantageous in this market, as it would allow for greater range, a larger area of influence, a static or mobile base station, and multiplexing.
		\item \textbf{Military applications} -- range and security are desirable features in military applications because they allow for secure surveillance and ground communications.
	\end{enumerate}

\section{\textsc{What capabilities does your product/device provide that are not available without this?}}
A phased array at the base station allows the drone to transmit a lower power and more robust signal over a larger distance because it tracks and dynamically tunes itself to receive the signal.  The base station also transmits lower power signals than a standard dipole because it can direct its beam toward the drone.  Additionally, the directivity of the phased array improves the security of the data sent between the drone and the user.  Using multiple frequency bands yields higher data throughput.  With this in mind, some potential use cases are as follows:
	\begin{itemize}
		\item Streaming high definition video at long ranges from the base station.
		\item Collecting weather and atmospheric data with high data rates.
		\item Relaying sensitive data from the drone to the base station.
		\item Connecting multiple drones to a single base station.
		\item Mounting base station to a vehicle - much easier with no moving parts and faster slew rate.
	\end{itemize}

\section{\textsc{What are the top project risks you perceive at this time?}}
	At this time, the risks we see are the following:
	\begin{itemize}
		\item The phased array and element design will be difficult, and will pose many challenges.  Much of our time will be dedicated to the antenna design and we will have to develop our schedule around this challenge.  Without a functioning antenna, the product will not have purpose. This can hopefully be mitigated with professional mentoring from First RF, and a few other local companies. 
		\item Each printed circuit board will cost between \$1,500 and \$2,500 to produce in order to have control of the board impedance.  This project has the potential to be quite costly.
		\item When designing the boards, RF Loss, noise, thermal conditions, and mechanical mounting interfaces will need to be taken into consideration.
		\item The power system of the project will need to be thought of very carefully.  To obtain a distance of $\sim$20 miles, a significant amount of power control is needed.
		\item Our antennas require specialized testing chambers in order to characterize their functionality (i.e.  gain, beam pattern, etc.).  We currently rely on First RF Corporation for use of their testing equipment.  If this falls through, we will require other means to move forward.
		\item Prof.  Argrow has proposed that a mechanical student may design the structure to house our phased array.  In order to give this student enough time to design this structure, we will have to finalize the majority of our design by the end of the semester.  
	\end{itemize}
\end{document}








