\documentclass[11pt]{article}
\usepackage{setspace}
\usepackage{hyperref}
\usepackage[utf8]{inputenc}
\usepackage[T1]{fontenc}
\usepackage{fullpage}
\usepackage{amsmath}
\usepackage{amssymb,amsfonts}
\usepackage[all,arc]{xy}
\usepackage{enumerate}
\usepackage{mathrsfs}
\usepackage{bbm}
\usepackage{graphicx}
\usepackage{subfiles}
\usepackage{setspace}
\usepackage{float}
\usepackage{subcaption}
\usepackage{verbatim}
\usepackage{listings}
\setlength{\footskip}{40pt}
\graphicspath{{Figures/}}
\numberwithin{figure}{section}
\setlength\parindent{0pt}

\begin{document}
%\doublespacing
\begin{center}
	\textsc{\Large Product Description}\\
	\hrulefill 
\end{center}

\section{\textsc{Team Members}}
	Matthew Arendall, Bryan DiLaura, Ariel Hoffman, Andrew Kee, Ryan Montoya, Jonathan Quinn
	
\section{\textsc{Team Name}}
	Phase Factor
	
\section{\textsc{What user problem is your product/device trying to solve?}}
	\textbf{High Priority Problems}  \\
	Typically, communication between a base station and an autonomous drone using dipole antennas is power intensive for two reasons: 
	\begin{enumerate}
		\item The base station needs to transmit over a large area because the drone could be in a wide range of places.
		\item The drone needs to transmit a signal powerful enough to be detected by a dipole on the base station which is not directionally sensitive and could be very far away.
	\end{enumerate}
	Currently, this is handled in a careful balancing act: designers seek to use as little power as possible while achieving the necessary signal to noise ratios. However, because a signal attenuates with the square of distance, this nonetheless can require a lot of energy.  More energy requires more weight. More battery weight inevitably means that some other features on the drone need to be cut. This could be speed, agility, or simply the removal of valuable electronic add-ons. From this perspective, the user is the drone and the problem is weight. \\
	
	In addition to power concerns, the range of the drone is severely limited by conventional means of communication.  A standard dipole radiates omnidirectionally, meaning that the majority of the signal is lost to the environment.  This seriously limits the operating range of the drone.  \\
	
	Furthermore, the omnidirectional transmission of data can be a security concern.  A ground receiver can potentially snoop on the communication channel and obtain sensitive information. This can be a problem for customers working a wartime scenario in which information can mean the difference between life and death.  \\

	\textbf{Medium Priority Problems} \\
	Standard communication protocols have limited bandwidth for communication and scientific data.  Applications that require large amounts of data consume significant bandwidth and therefore are limited by the carrier frequency band.  Operating on a single channel requires multiplexing data and command/control signals, further increasing the complexity and bandwidth requirements of the system. This can be a problem for customers interested in transmitting telemetry and wifi to their drone at the same time. \\
	
	Another problem for people who want to work with high reliability drones is that sometimes links go down. If an antenna stops working, it is important to be able to reestablish a link some other way. \\
	
	\textbf{Low Priority Problems} \\
	If a user wants to be able to manage multiple drones at once, one option is to use multiple antennas. This allows people to send the necessary bandwidth of data. However, this solution can be expensive because it costs a lot to make multiple antennas and create independent tracking systems. An alternative option is to use a mechanical antenna director and write firmware to allow the antenna to multiplex. This is a problem because it is unreliable and slow due to its dependence on mechanical components. Both of these problems are prohibitive to users trying to do research.
	

	
\section{\textsc{What is our proposed solution?}}
	\textbf{High Priority Solutions}  \\
	We propose using a stacked phased array to solve these problems.  A phased array is an array of dipoles that allows us to create a highly directive beam pattern.  This allows the user to transmit over a larger distance using less power when compared to a single dipole transmitting over the same distance.  A stacked phased array concurrently uses two or more frequency isolated phased array systems.  We propose using both a 900 MHz and 2.4 GHz in our stacked phased array system. \\
	
	The range of a phased array is significantly longer than a dipole due to directivity.  This will enable the base station to receive and transmit to the drone antenna at long ranges even if the drone only supports a standard dipole or low power signal. Furthermore, the phased array is capable of tuning itself to receive a certain phase more effectively. Assuming the phased array can keep a lock on the drone, this means that it will be able to receive lower power signals from the drone, which addresses the battery problem.  \\
	
	To address the concern of data security, the high directivity of the signal allows it to point to a very small area, thereby reducing the possibility that another person would be able to intercept the signal.\\
	
	\textbf{Medium Priority Solutions}
	The data and telemetry multiplexing problem can be addressed by using two isolated frequency bands. Two frequency bands gives the system gains a much greater bandwidth.  Additionally, using two isolated bands allows separation of data and command/control.  This reduces the complexity.  We propose to use a 2.4 GHz channel for data as it has a greater bandwidth and is well suited for large data throughput, and to use a 900 MHz channel for command/control as it is long range. \\
	
	\textbf{Low Priority Solutions}
	In order to solve the multiple drone problem, we propose using a high speed microcontroller to manage our phased arrays. Because the phased array is capable of redirecting the beam electronically, it solves the mechanical issues. Because we will be using a high frequency chip, we will be able to keep a lock on multiple drones at once. Finally, because we will be able to package this functionality into a single product, it will be cheaper, more portable, and more maintainable.

\section{\textsc{Who needs this? Who will benefit from this?}}
There are three major beneficiaries: 
	\begin{enumerate}
		\item \textbf{Researchers} -- the drone is intended as a lightweight surveillance system. The drone needs to be able to travel significant distances in potentially hard to reach areas and transmit data back to the base. Our phased array project helps improve the functionality of the drone by increasing the range and reducing the weight of the drone. The drone can be used for weather analysis, video streaming, or relaying information.
		\item \textbf{The autonomous vehicle market} -- most people within the market are using standard dipole antennas and are dealing with all of the shortcomings as a result (discussed earlier). A phased array would be advantageous in this market, as it would allow for greater range, a larger area of influence, a static or mobile base station, and multiplexing.
		\item \textbf{Military applications} -- range and security are desirable features in military applications because they allow for secure surveillance and ground communications.
	\end{enumerate}

\section{\textsc{What capabilities does your product/device provide that are not available without this?}}
A phased array at the base station allows the drone to transmit a lower power signal over a larger distance because it tracks and dynamically tunes itself to receive the signal. The base station also transmits lower power signals because it can direct its beam toward the drone. Additionally, the directivity of the phased array improves the security of the data sent between the drone and the user.  Using multiple frequency bands yields higher data throughput.  With this in mind, some potential use cases are as follows:
	\begin{itemize}
		\item Streaming high definition video at long ranges from the base station.
		\item Collecting weather and atmospheric data with high data rates.
		\item Relaying sensitive data from the drone to the base station.
	\end{itemize}

\section{\textsc{What are the top project risks you perceive at this time?}}
	At this time, the risks we see are the following:
	\begin{itemize}
		\item RF design is sensitive to small design changes, especially when antennas are concerned.  Much of our time will be dedicated to the antenna design if we decide to go down this route.
		\item We are leveraging existing hardware designs in our project.  This may pose a problem if the existing designs do not meet our specifications.
		\item We will need develop a communications interface between the base station and the drone.  Developing the firmware could prove more time-costly than we plan.
		\item Because we are designing a project dealing with antennas, each board will cost between \$1,500 and \$2,000 to produce. Because every small mistake is very costly, the project itself will be very costly. There is a small possibility that this project might not meet standards and nonetheless cost a lot.
	\end{itemize}
\end{document}








